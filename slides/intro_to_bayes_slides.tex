% Copyright 2003--2007 by Till Tantau
% Copyright 2010 by Vedran Mileti\'c
% Copyright 2012 by Jeffrey Arnold
%
% This file may be distributed and/or modified
%
% 1. under the LaTeX Project Public License and/or
% 2. under the GNU Free Documentation License.
%
% See the file doc/licenses/LICENSE for more details.
%
% Slightly modified for the beamerthemesolarized to turn
% it into a jinja2 template.


%--------------------------------------------------------------------------------------
\documentclass[hyperref={draft}]{beamer}

\usecolortheme[dark]{solarized}
\setbeamertemplate{navigation symbols}{}

\beamertemplatetransparentcovered

\usepackage{times}

\title{An Introduction to Bayesian Statistics}

\author{Simon Thornewill von Essen}
\institute{Data Analyst\\ Goodgame Studios}
\date{Data Science Meetup Hamburg -- 2019-08-22}

\begin{document}

\begin{frame}
  \titlepage
\end{frame}

\begin{frame}<1>
  \tableofcontents
\end{frame}

\begin{frame}<1>
  $$\theta = (A^TA)^{-1}A^Ty$$
\end{frame}

\section{Results}
\subsection{Proof of the Main Theorem}

\begin{frame}<2>
  \frametitle{There Is No Largest Prime Number}
  \framesubtitle{The proof uses \textit{reductio ad absurdum}.}

  \begin{theorem}
    There is no largest prime number.
  \end{theorem}
  \begin{proof}
    \begin{enumerate}
      % The strange way of typesetting math is to minimize font usage
      % in order to keep the file sizes of the examples small.
    \item<1-| alert@1> Suppose $p$ were the largest prime number.
    \item<2-> Let $q$ be the product of the first $p$ numbers.
    \item<3-> Then $q$\;+\,$1$ is not divisible by any of them.
    \item<1-> Thus $q$\;+\,$1$ is also prime and greater than $p$.\qedhere
    \end{enumerate}
  \end{proof}
\end{frame}

\end{document}
